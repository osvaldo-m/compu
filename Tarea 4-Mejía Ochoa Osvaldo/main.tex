\documentclass[letterpaper,pt12]{article}
\usepackage[utf8]{inputenc}
\usepackage{amsmath, amssymb}
\usepackage{latexsym}
\usepackage{mathrsfs}
\usepackage{xcolor}
\usepackage{tcolorbox}
\usepackage{fancyhdr}




 
 
 
 
\begin{document}

\textit{\textbf{\huge{fOrMuLaS pArA aPrObAr :C}}} 

\pagestyle{fancy}

\fancyhf{}
\chead{Física}
\lfoot{Mejía Ochoa Osvaldo}
 \rfoot{\thepage}
 
\section*{Física}

\begin{itemize}
\item [$\circledast$]{Distancia.}\\
Esta fórmula se utiliza para medir la distancia en física;"a" es aceleración, "v" es Velocidad y "t" es tiempo.
\begin{equation*}
    d=v\cdot t
\end{equation*}

\item [$\circledast$]{Aceleración.}\\
Esta fórmula se utiliza para calcular la aceleración;"Vi" es la velocidad inicial, "Vf" es la velocidad final y "t" es tiempo.
\begin{equation*}
     a=\frac{Vf-Vi}{t}
\end{equation*}

\item [$\circledast$]{Caida libre.}\\
Esta fórmula se utiliza para calcular la caida libre de un objeto desde cierta altura;"y" es la altura, "Vi" es la velocidad inicial, "Vf" es la velocidad final y "t" es tiempo.
\begin{equation*}
    y=\frac{Vf+Vi}{2}t
\end{equation*}

\item [$\circledast$]{Trabajo}\\
El trabajo es calcular la cantidad de energía empleada para cierta cosa;"w" es trabajo, "f" es la fuerza y "d" es la distancia.
\begin{equation*}
    w=f\cdot d
\end{equation*}

\item [$\circledast$]{Fuerza}\\
La fuerza empleada es la fuerza de gravedad que actúa sobre algo;"f" es fuerza, "m" es masa y "g" es la constante de la aceleración de la gravedad. 
\begin{equation*}
    f=m\cdot g
\end{equation*}

\item [$\bigstar$]{Energía cinética.}\\
La energía de movimiento que tiene un cuerpo se conoce como energía cinética;"ec" es la energía cinética, "m" es la masa, "v" es la velocidad.
\begin{equation*}
    ec=\frac{m\cdot v^2}{2}
\end{equation*}

\item [$\bigstar$]{Energía potencial.}\\
La energía potencial es la energía que tiene almacenada un cuerpo debido a su posición dentro de un sistema;"ep" es energía potencial, "m" es la masa, "g" es la constante de la aceleración de la gravedad y "h" es la altura.
\begin{equation*}
    ep=m\cdot g\cdot h
\end{equation*}

\item [$\bigstar$]{2da ley de Newton}\\
La segunda ley de Newton, llamada ley fundamental o principio fundamental de la dinámica, plantea que un cuerpo se acelera si se le aplica una fuerza;"f" es fuerza, "m" es masa y "a" aceleración.
\begin{equation*}
    \textcolor{red}{f=m\cdot a}
\end{equation*}

\item [$\bigstar$]{Cantidad de movimiento.}\\
Es una magnitud física fundamental de tipo vectorial que describe el movimiento de un cuerpo en cualquier teoría mecánica;"p" es cantidad de movimiento, "m" es masa del cuerpo y "v" es velocidad.
\begin{equation*}
    p=m\cdot v
\end{equation*}

\item [$\bigstar$]{Densidad.}\\
Es el grado de compactación de la masa de materia;"p" es densidad,"m" es masa y "V" es volumen.
\begin{equation*}
    p=\frac{m}{V}
\end{equation*}
\end{itemize}

\pagestyle{fancy}

\fancyhf{}
\chead{Matemáticas}
\lfoot{Mejía Ochoa Osvaldo}
 \rfoot{\thepage}

\section*{Matemáticas}
\begin{itemize}
\item [$\ast$]{Término general de una sucesión}\\
Se puede calcular cualquier término de una sucesión aritmética calculando todos los valores intermedios o mediante la fórmula del término general de una sucesión aritmética.
\begin{equation*}
    U_n=U_0+n\cdot d
\end{equation*}

\item[$\ast$]{Distancia entre dos puntos.}\\
Sirve para calcular la distancia entre dos puntos en un plano cartesiano.
\begin{equation*}
    d=\sqrt{(x_2-x_1)^2+(y_2-y_1)^2}
\end{equation*}

\item[$\ast$]{Teorema de pitágoras.}\\
En esta se describe la relación entre los lados de un triángulo rectángulo en una superficie plana.
\begin{equation*}
    a^2+b^2=c^2
\end{equation*}

\item[$\ast$]{Base del cálculo}\\
Esta ecuación ayuda a comprender el cambio de las funciones cuando sus variables cambian.
\begin{equation*}
    \frac{df}{dt}=\lim_{h \to 0}\frac{f(t+h)-f(t)}{h}
\end{equation*}

\item[$\ast$]{Logaritmos.}\\
Sirven para calcular numeros muy grandes.
\begin{equation*}
    \log xy=\log x+\log y
\end{equation*}

\item[$\ast$]{Identidad de Euler.}\\
Sirve para relacionar la trigonometría con el análisis matemático.
\begin{equation*}
    e^{i\pi}+1=0
\end{equation*}

\item[$\ast$]{Área de un triángulo}\\
Sirve para identificar el área que abarca un triángulo según sus dimensiones.
\begin{equation*}
    a=\frac{b\cdot h}{2}
\end{equation*}

\item[$\ast$]{Área de un rectángulo}\\
Sirve para identificar el área que abarca un rectángulo según sus dimensiones.
\begin{equation*}
    a=b\cdot h
\end{equation*}

\item[$\ast$]{Área de un rombo}\\
Sirve para identificar el área que abarca un rombo según sus dimensiones; "D" es diagonal mayor y "d" es diagonal menor.
\begin{equation*}
    a=\frac{D\cdot d}{2}
\end{equation*}

\item[$\ast$]{Fórmula general}\\
Esta permite obtener el valor de una incógnita en distintos casos particulares.
\begin{equation*}
     \textcolor{red}{x=\frac{-b \pm \sqrt{b^{2}- 4ac}} {2a}}
\end{equation*}
\end{itemize}   
   
\pagestyle{fancy}

\fancyhf{}
\chead{Trigonometría}   
\lfoot{Mejía Ochoa Osvaldo}
 \rfoot{\thepage}
   
\section*{Trigonometría}
Son de gran importancia para la medición de ángulos y triángulos, todas las formulas siguientes son parte de ellas.
\begin{itemize}
\item [$\diamond$]{Seno.}
\begin{equation*}
    \textcolor{red}{sen B=\frac{cateto\quad opuesto}{hipotenusa}}
\end{equation*}  

\item [$\diamond$]{Coseno.}
\begin{equation*}
  cosB=\frac{cateto\quad adyacente}{hipotenusa}  
\end{equation*} 

\item [$\diamond$]{Tangente.}
\begin{equation*}
    tanB=\frac{cateto\quad opuesto}{cateto\quad adyacente}
\end{equation*} 

\item [$\diamond$]{Cotangente.}
\begin{equation*}
    cotB=\frac{1}{tanB}
\end{equation*} 

\item [$\diamond$]{Secante.}
\begin{equation*}
    secB=\frac{1}{cosB}
\end{equation*} 
    
    
    
    
\end{itemize}









\end{document}